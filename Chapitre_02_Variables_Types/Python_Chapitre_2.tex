
\documentclass[12pt, a4paper]{article}

% ============================================================
%  PACKAGES
% ============================================================
\usepackage[utf8]{inputenc}
\usepackage[T1]{fontenc}
\usepackage[french]{babel}
\usepackage{geometry}
\usepackage{graphicx}
\usepackage{xcolor}
\usepackage{tikz}
\usepackage{mdframed}
\usepackage{listings}
\usepackage{fontawesome5}
\usepackage{fancyhdr}
\usepackage{titlesec}
\usepackage{enumitem}
\usepackage{lmodern}
\usepackage{microtype}
\usepackage{hyperref}
\usepackage{tcolorbox}
\tcbuselibrary{skins, breakable, listings}

\geometry{
  top=2.5cm,
  bottom=2.5cm,
  left=2.5cm,
  right=2.5cm
}

% ============================================================
%  COULEURS
% ============================================================
\definecolor{bleuPrimaire}{HTML}{1A3C6E}
\definecolor{bleuSecondaire}{HTML}{2E6DBD}
\definecolor{bleuClair}{HTML}{EAF2FB}
\definecolor{bleuAccent}{HTML}{4A90D9}
\definecolor{grisTexte}{HTML}{2C3E50}
\definecolor{grisClair}{HTML}{F4F6F9}
\definecolor{vertCode}{HTML}{27AE60}
\definecolor{orangeNote}{HTML}{E67E22}
\definecolor{rougeImportant}{HTML}{E74C3C}
\definecolor{codebg}{HTML}{1E2A3A}
\definecolor{codetext}{HTML}{ECF0F1}

% ============================================================
%  STYLE DE CODE
% ============================================================
\lstdefinestyle{pythonstyle}{
  language=Python,
  backgroundcolor=\color{codebg},
  basicstyle=\ttfamily\small\color{codetext},
  keywordstyle=\color{bleuAccent}\bfseries,
  stringstyle=\color{vertCode},
  commentstyle=\color{gray}\itshape,
  numberstyle=\tiny\color{gray},
  numbers=left,
  numbersep=8pt,
  stepnumber=1,
  showstringspaces=false,
  breaklines=true,
  frame=none,
  tabsize=4,
  captionpos=b,
  morekeywords={print, input, type, int, float, str, bool, True, False}
}

% ============================================================
%  EN-TÊTE ET PIED DE PAGE
% ============================================================
\pagestyle{fancy}
\fancyhf{}
\renewcommand{\headrulewidth}{0pt}
\fancyhead[L]{%
  \begin{tikzpicture}[remember picture, overlay]
    \fill[bleuPrimaire] (0,0) rectangle (\paperwidth, 1.1cm);
  \end{tikzpicture}%
  \color{white}\small\textbf{Cours Python — Mon Apprentissage}
}
\fancyhead[R]{\color{white}\small\textbf{Chapitre 2 : Variables et Types de Données}}
\fancyfoot[C]{%
  \begin{tikzpicture}[remember picture, overlay]
    \fill[bleuClair] (0,0) rectangle (\paperwidth, 0.7cm);
    \draw[bleuSecondaire, line width=1pt] (0, 0.7cm) -- (\paperwidth, 0.7cm);
  \end{tikzpicture}%
  \color{bleuPrimaire}\small\thepage
}

% ============================================================
%  STYLE DES TITRES
% ============================================================
\titleformat{\section}{%
  \color{bleuPrimaire}\Large\bfseries
}{}{0em}{}[\vspace{-0.3em}\textcolor{bleuSecondaire}{\rule{\linewidth}{1.5pt}}]

\titleformat{\subsection}{%
  \color{bleuSecondaire}\large\bfseries
}{}{0em}{}

\titlespacing{\section}{0pt}{1.5em}{0.8em}
\titlespacing{\subsection}{0pt}{1em}{0.5em}

% ============================================================
%  BOÎTES PERSONNALISÉES
% ============================================================
\newtcolorbox{boiteRetenir}{
  enhanced, breakable,
  colback=bleuClair, colframe=bleuSecondaire,
  arc=4pt, boxrule=1.5pt,
  left=10pt, right=10pt, top=8pt, bottom=8pt,
  title={\faLightbulb\quad À retenir},
  fonttitle=\bfseries\color{bleuPrimaire},
  attach boxed title to top left={yshift=-2mm, xshift=5mm},
  boxed title style={colback=bleuClair, colframe=bleuSecondaire}
}

\newtcolorbox{boiteNote}{
  enhanced, breakable,
  colback=orange!8, colframe=orangeNote,
  arc=4pt, boxrule=1.5pt,
  left=10pt, right=10pt, top=8pt, bottom=8pt,
  title={\faExclamationTriangle\quad Note importante},
  fonttitle=\bfseries\color{orangeNote},
  attach boxed title to top left={yshift=-2mm, xshift=5mm},
  boxed title style={colback=orange!8, colframe=orangeNote}
}

\newtcolorbox{boiteErreur}{
  enhanced, breakable,
  colback=red!5, colframe=rougeImportant,
  arc=4pt, boxrule=1.5pt,
  left=10pt, right=10pt, top=8pt, bottom=8pt,
  title={\faBug\quad Erreur fréquente},
  fonttitle=\bfseries\color{rougeImportant},
  attach boxed title to top left={yshift=-2mm, xshift=5mm},
  boxed title style={colback=red!5, colframe=rougeImportant}
}

\newtcblisting{codeblock}[1]{
  enhanced, breakable,
  listing only,
  listing style=pythonstyle,
  colback=codebg, colframe=bleuSecondaire,
  arc=6pt, boxrule=1pt,
  title={\faCode\quad #1},
  fonttitle=\bfseries\color{white}\small,
  attach boxed title to top left={yshift=-2mm, xshift=5mm},
  boxed title style={colback=bleuSecondaire, colframe=bleuSecondaire},
  left=6pt, right=6pt, top=8pt, bottom=6pt
}

% ============================================================
%  PAGE DE TITRE
% ============================================================
\begin{document}

\begin{titlepage}
  \begin{tikzpicture}[remember picture, overlay]
    \fill[bleuPrimaire] (current page.north west) rectangle
      ([yshift=-10cm] current page.north east);
    \fill[bleuSecondaire] ([yshift=-10cm] current page.north west) rectangle
      ([yshift=-11cm] current page.north east);
    \fill[grisClair] ([yshift=-11cm] current page.north west) rectangle
      (current page.south east);
    \fill[bleuAccent, opacity=0.15]
      ([xshift=12cm, yshift=-5cm] current page.north west) circle (5cm);
    \fill[bleuAccent, opacity=0.08]
      ([xshift=15cm, yshift=-3cm] current page.north west) circle (3cm);
  \end{tikzpicture}

  \vspace*{2cm}

  \begin{center}
    {\fontsize{14}{16}\selectfont\color{bleuClair}
      \textbf{MON APPRENTISSAGE — SERIES PYTHON}}\\[0.5cm]
    {\fontsize{42}{48}\selectfont\color{white}\textbf{Python}}\\[0.2cm]
    {\fontsize{20}{24}\selectfont\color{bleuAccent}
      \textbf{Du zéro à l'expert}}\\[0.3cm]
    \textcolor{white!70}{\rule{8cm}{0.5pt}}\\[0.5cm]
    {\fontsize{16}{20}\selectfont\color{white}
      \textbf{Chapitre 2 — Variables et Types de Données}}\\[0.2cm]
    {\fontsize{13}{16}\selectfont\color{bleuClair}
      \textit{Comprendre comment Python stocke l'information}}
  \end{center}

  \vspace{3.5cm}

  \begin{center}
    \begin{tikzpicture}
      \fill[bleuClair, rounded corners=8pt] (0,0) rectangle (12cm, 3.2cm);
      \draw[bleuSecondaire, line width=1.5pt, rounded corners=8pt]
        (0,0) rectangle (12cm, 3.2cm);
      \node[anchor=west] at (0.4, 2.5)
        {\color{bleuPrimaire}\small\textbf{Auteur :}};
      \node[anchor=west] at (0.4, 1.9)
        {\color{grisTexte}\large\textbf{KETOTSA AMÉVI CLAUDE}};
      \node[anchor=west] at (0.4, 1.2)
        {\color{bleuPrimaire}\small\textbf{Date de publication :}};
      \node[anchor=west] at (0.4, 0.6)
        {\color{grisTexte}\small\today};
    \end{tikzpicture}
  \end{center}

  \vfill
  \begin{center}
    \color{bleuSecondaire}\small\faLinkedin\quad
    \textit{Publié dans le cadre de mon parcours d'apprentissage Python}
  \end{center}
\end{titlepage}

% ============================================================
%  TABLE DES MATIÈRES
% ============================================================
\newpage
\tableofcontents
\newpage

% ============================================================
%  CHAPITRE 2
% ============================================================

\section*{\faPython\quad Chapitre 2 — Variables et Types de Données}
\addcontentsline{toc}{section}{Chapitre 2 — Variables et Types de Données}

\vspace{0.3cm}

\begin{boiteRetenir}
Dans ce chapitre, je découvre comment Python \textbf{stocke et manipule l'information}.
Les variables sont les briques fondamentales de tout programme.
Comprendre les types de données, c'est comprendre comment Python pense.
\end{boiteRetenir}

\vspace{0.5cm}

% -----------------------------------------------------------
\subsection{Qu'est-ce qu'une variable ?}

Une variable, c'est comme une \textbf{boîte avec une étiquette}. On donne un nom à cette boîte, on y met une valeur, et on peut la retrouver plus tard grâce à son nom.

En Python, créer une variable est d'une simplicité totale — pas besoin de déclarer le type, Python s'en charge automatiquement.

\begin{codeblock}{Mes premières variables}
# Creer des variables aussi simple que ca
prenom = "Alice"
age = 25
taille = 1.68
est_etudiant = True

# Afficher les valeurs
print(prenom)        # Alice
print(age)           # 25
print(taille)        # 1.68
print(est_etudiant)  # True
\end{codeblock}

\begin{boiteNote}
En Python, on n'écrit \textbf{jamais} le type de la variable avant son nom.
Pas de \texttt{int age = 25} comme en Java ou C++.
On écrit simplement \texttt{age = 25} et Python comprend tout seul.
\end{boiteNote}

% -----------------------------------------------------------
\subsection{Les règles de nommage des variables}

Nommer une variable n'est pas totalement libre. Python impose quelques règles :

\begin{center}
\begin{tikzpicture}
  % Colonne AUTORISÉ
  \fill[vertCode!10, rounded corners=6pt] (0,0) rectangle (6.5, 5.5);
  \draw[vertCode, line width=1.5pt, rounded corners=6pt] (0,0) rectangle (6.5,5.5);
  \node[font=\bfseries\small, text=vertCode] at (3.25, 5.1)
    {\faCheckCircle\quad AUTORISÉ};
  \node[font=\ttfamily\small, text=grisTexte, align=left] at (3.25, 3.8)
    {mon\_prenom};
  \node[font=\ttfamily\small, text=grisTexte, align=left] at (3.25, 3.1)
    {age2};
  \node[font=\ttfamily\small, text=grisTexte, align=left] at (3.25, 2.4)
    {\_variable\_privee};
  \node[font=\ttfamily\small, text=grisTexte, align=left] at (3.25, 1.7)
    {nomDeVariable};
  \node[font=\ttfamily\small, text=grisTexte, align=left] at (3.25, 1.0)
    {CONSTANTE};

  % Colonne INTERDIT
  \fill[red!8, rounded corners=6pt] (7,0) rectangle (13.5, 5.5);
  \draw[rougeImportant, line width=1.5pt, rounded corners=6pt]
    (7,0) rectangle (13.5,5.5);
  \node[font=\bfseries\small, text=rougeImportant] at (10.25, 5.1)
    {\faTimesCircle\quad INTERDIT};
  \node[font=\ttfamily\small, text=grisTexte] at (10.25, 3.8)
    {2age \quad (commence par un chiffre)};
  \node[font=\ttfamily\small, text=grisTexte] at (10.25, 3.1)
    {mon-prenom \quad (tiret interdit)};
  \node[font=\ttfamily\small, text=grisTexte] at (10.25, 2.4)
    {mon prenom \quad (espace interdit)};
  \node[font=\ttfamily\small, text=grisTexte] at (10.25, 1.7)
    {class \quad (mot réservé Python)};
  \node[font=\ttfamily\small, text=grisTexte] at (10.25, 1.0)
    {prénom \quad (accents déconseillés)};
\end{tikzpicture}
\end{center}

% -----------------------------------------------------------
\subsection{Les 4 types de base en Python}

Python possède 4 types fondamentaux que tout développeur doit maîtriser :

\vspace{0.5cm}

\begin{center}
\begin{tikzpicture}
  \foreach \type/\desc/\exemple/\couleur/\pos in {
    {int}/{Nombres entiers}/{42, -10, 0}/{bleuPrimaire}/0,
    {float}/{Nombres décimaux}/{3.14, -0.5, 1.0}/{bleuSecondaire}/3.5,
    {str}/{Chaînes de texte}/{"Bonjour", "Python"}/{vertCode}/7,
    {bool}/{Vrai ou Faux}/{True, False}/{orangeNote}/10.5
  }{
    \fill[\couleur!15, rounded corners=6pt]
      (\pos, 0) rectangle (\pos+3, 3.5);
    \draw[\couleur, line width=1.5pt, rounded corners=6pt]
      (\pos, 0) rectangle (\pos+3, 3.5);
    \node[font=\bfseries, text=\couleur] at (\pos+1.5, 3.1)
      {\type};
    \node[font=\small, text=grisTexte, align=center] at (\pos+1.5, 2.3)
      {\desc};
    \draw[\couleur!40, dashed] (\pos+0.2, 1.8) -- (\pos+2.8, 1.8);
    \node[font=\ttfamily\tiny, text=grisTexte, align=center]
      at (\pos+1.5, 1.1) {\exemple};
  }
\end{tikzpicture}
\end{center}

\vspace{0.5cm}

\begin{codeblock}{Les 4 types fondamentaux en action}
# int — nombre entier
nombre_entier = 42
annee = 2024
temperature_negative = -15

# float — nombre decimal
pi = 3.14159
prix = 19.99
note = -0.5

# str — chaine de caracteres
prenom = "Alice"
message = 'Bonjour tout le monde !'
phrase = "J'apprends Python aujourd'hui"

# bool — booleen (vrai ou faux)
est_connecte = True
a_paye = False
majeur = True

# Verifier le type avec type()
print(type(nombre_entier))   # <class 'int'>
print(type(pi))              # <class 'float'>
print(type(prenom))          # <class 'str'>
print(type(est_connecte))    # <class 'bool'>
\end{codeblock}

% -----------------------------------------------------------
\subsection{Le type int — Les entiers}

Le type \texttt{int} représente tous les nombres entiers, positifs ou négatifs, sans limite de taille en Python.

\begin{codeblock}{Travailler avec les entiers}
# Declarations
population_monde = 8_000_000_000   # underscore pour lisibilite
temperature_lune = -173
etage = 0

# Operations de base
a = 10
b = 3

print(a + b)   # Addition      → 13
print(a - b)   # Soustraction  → 7
print(a * b)   # Multiplication → 30
print(a // b)  # Division entière → 3
print(a % b)   # Modulo (reste) → 1
print(a ** b)  # Puissance     → 1000
\end{codeblock}

\begin{boiteRetenir}
En Python, on peut utiliser des \textbf{underscores} dans les grands nombres pour les rendre plus lisibles : \texttt{1\_000\_000} est identique à \texttt{1000000}.
\end{boiteRetenir}

% -----------------------------------------------------------
\subsection{Le type float — Les décimaux}

Le type \texttt{float} représente les nombres à virgule flottante (décimaux).

\begin{codeblock}{Travailler avec les décimaux}
# Declarations
prix_cafe = 2.50
taux_tva = 0.20
gravite = 9.81

# Calcul du prix TTC
prix_ht = 100.0
prix_ttc = prix_ht * (1 + taux_tva)
print(prix_ttc)   # 120.0

# Attention à la precision des floats !
print(0.1 + 0.2)  # 0.30000000000000004 (pas 0.3 !)

# Solution : arrondir avec round()
print(round(0.1 + 0.2, 2))  # 0.3
\end{codeblock}

\begin{boiteErreur}
Les \texttt{float} ne sont pas toujours précis à cause de la représentation binaire.
\texttt{0.1 + 0.2} ne donne pas exactement \texttt{0.3} en Python.
Utilisez \texttt{round()} pour arrondir vos résultats.
\end{boiteErreur}

% -----------------------------------------------------------
\subsection{Le type str — Les chaînes de caractères}

Le type \texttt{str} (string) représente du texte. On peut utiliser des guillemets simples ou doubles.

\begin{codeblock}{Travailler avec les chaînes de caractères}
# Declarations
prenom = "Alice"
ville = 'Paris'
message = "J'habite a Paris"   # guillemets doubles si apostrophe

# Concatenation (assembler des chaines)
nom_complet = "Alice" + " " + "Dupont"
print(nom_complet)   # Alice Dupont

# f-string — la methode moderne et recommandee
age = 25
presentation = f"Je m'appelle {prenom} et j'ai {age} ans."
print(presentation)  # Je m'appelle Alice et j'ai 25 ans.

# Longueur d'une chaine
print(len(prenom))   # 5

# Acceder a un caractere (index commence a 0)
print(prenom[0])     # A
print(prenom[-1])    # e (dernier caractere)

# Mettre en majuscules / minuscules
print(prenom.upper())   # ALICE
print(prenom.lower())   # alice
\end{codeblock}

% -----------------------------------------------------------
\subsection{Le type bool — Les booléens}

Le type \texttt{bool} n'a que deux valeurs possibles : \texttt{True} ou \texttt{False}. Il est fondamental pour les conditions et la logique.

\begin{codeblock}{Travailler avec les booléens}
# Declarations
est_majeur = True
est_connecte = False

# Les comparaisons retournent des booleens
age = 20
print(age >= 18)    # True
print(age == 15)    # False
print(age != 18)    # True

# Operateurs logiques
a = True
b = False
print(a and b)   # False (les deux doivent etre True)
print(a or b)    # True  (au moins un doit etre True)
print(not a)     # False (inverse la valeur)

# Exemple concret
a_un_compte = True
a_paye = False
peut_acceder = a_un_compte and a_paye
print(peut_acceder)   # False
\end{codeblock}

% -----------------------------------------------------------
\subsection{Le typage dynamique de Python}

Python est un langage à \textbf{typage dynamique} : une variable peut changer de type en cours de programme. Python détecte automatiquement le type selon la valeur assignée.

\begin{codeblock}{Le typage dynamique en action}
# Une variable peut changer de type
x = 10
print(type(x))   # <class 'int'>

x = 3.14
print(type(x))   # <class 'float'>

x = "Bonjour"
print(type(x))   # <class 'str'>

x = True
print(type(x))   # <class 'bool'>

# Python s'adapte à chaque fois — pas d'erreur !
\end{codeblock}

\begin{boiteNote}
Le typage dynamique est une \textbf{force} de Python — il rend le code plus rapide à écrire. Mais il demande aussi d'être rigoureux pour ne pas mélanger les types sans le vouloir.
\end{boiteNote}

% -----------------------------------------------------------
\subsection{La conversion de types — Casting}

On peut convertir un type en un autre grâce aux fonctions de conversion :

\begin{codeblock}{Conversion de types (casting)}
# str → int
age_texte = "25"
age_nombre = int(age_texte)
print(age_nombre + 5)   # 30

# int → float
nombre = 10
print(float(nombre))    # 10.0

# float → int (attention : tronque la decimale !)
pi = 3.99
print(int(pi))          # 3  (pas 4 !)

# int → str
annee = 2024
message = "Nous sommes en " + str(annee)
print(message)          # Nous sommes en 2024

# Recuperer une saisie utilisateur (toujours du str !)
# age = input("Ton age : ")    → renvoie toujours une str
# age = int(input("Ton age : ")) → convertir en int
\end{codeblock}

\begin{boiteErreur}
\texttt{input()} retourne \textbf{toujours une chaîne de caractères} (str), même si l'utilisateur tape un nombre.
Il faut toujours convertir avec \texttt{int()} ou \texttt{float()} si vous voulez faire des calculs.
\end{boiteErreur}

% -----------------------------------------------------------
\subsection{Les variables multiples}

Python permet d'assigner plusieurs variables en une seule ligne, ce qui est très élégant :

\begin{codeblock}{Assignations multiples}
# Assigner la meme valeur a plusieurs variables
x = y = z = 0
print(x, y, z)   # 0 0 0

# Assigner plusieurs valeurs en une ligne
prenom, nom, age = "Alice", "Dupont", 25
print(prenom)   # Alice
print(nom)      # Dupont
print(age)      # 25

# Echanger deux variables (unique a Python !)
a = 10
b = 20
a, b = b, a
print(a, b)   # 20 10
\end{codeblock}

\begin{boiteRetenir}
Échanger deux variables avec \texttt{a, b = b, a} est une syntaxe \textbf{unique à Python}. Dans d'autres langages, il faut une variable temporaire. Python fait ça en une ligne. Élégant.
\end{boiteRetenir}

% -----------------------------------------------------------
\subsection{Résumé du Chapitre 2}

\begin{tcolorbox}[
  enhanced, breakable,
  colback=bleuPrimaire,
  colframe=bleuPrimaire,
  arc=8pt,
  left=15pt, right=15pt, top=10pt, bottom=10pt
]
  \color{white}
  \textbf{\large\faClipboardList\quad Ce que j'ai appris dans ce chapitre :}

  \vspace{0.5em}
  \begin{itemize}[leftmargin=1.5em, itemsep=6pt]
    \item[\textcolor{bleuAccent}{\faCheckCircle}]
      Une variable est une boîte nommée qui stocke une valeur.
    \item[\textcolor{bleuAccent}{\faCheckCircle}]
      Les 4 types de base : \texttt{int}, \texttt{float}, \texttt{str}, \texttt{bool}.
    \item[\textcolor{bleuAccent}{\faCheckCircle}]
      Python est à typage dynamique — il détecte le type automatiquement.
    \item[\textcolor{bleuAccent}{\faCheckCircle}]
      \texttt{type()} permet de connaître le type d'une variable à tout moment.
    \item[\textcolor{bleuAccent}{\faCheckCircle}]
      Le casting (\texttt{int()}, \texttt{float()}, \texttt{str()}) permet de convertir les types.
    \item[\textcolor{bleuAccent}{\faCheckCircle}]
      \texttt{input()} retourne toujours une \texttt{str} — toujours convertir pour les calculs.
    \item[\textcolor{bleuAccent}{\faCheckCircle}]
      Python permet d'échanger deux variables en une ligne : \texttt{a, b = b, a}.
  \end{itemize}

  \vspace{0.5em}
  \textit{\textcolor{bleuClair}{"Les données sont la matière première du XXIe siècle.
  Savoir les manipuler, c'est savoir construire."}}
\end{tcolorbox}

% -----------------------------------------------------------
\vspace{1cm}
\begin{center}
  \begin{tikzpicture}
    \fill[bleuClair, rounded corners=6pt] (0,0) rectangle (16, 1.5);
    \draw[bleuSecondaire, line width=1pt, rounded corners=6pt]
      (0,0) rectangle (16, 1.5);
    \node[font=\small, text=bleuPrimaire] at (8, 0.75) {%
      \faLinkedin\quad \textbf{KETOTSA AMÉVI CLAUDE} — Mon apprentissage Python
      \quad|\quad Chapitre 2 sur 34
    };
  \end{tikzpicture}
\end{center}

\end{document}