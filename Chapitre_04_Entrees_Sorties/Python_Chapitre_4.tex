
\documentclass[12pt, a4paper]{article}

% ============================================================
%  PACKAGES
% ============================================================
\usepackage[utf8]{inputenc}
\usepackage[T1]{fontenc}
\usepackage[french]{babel}
\usepackage{geometry}
\usepackage{graphicx}
\usepackage{xcolor}
\usepackage{tikz}
\usepackage{mdframed}
\usepackage{listings}
\usepackage{fontawesome5}
\usepackage{fancyhdr}
\usepackage{titlesec}
\usepackage{enumitem}
\usepackage{lmodern}
\usepackage{microtype}
\usepackage{hyperref}
\usepackage{tcolorbox}
\tcbuselibrary{skins, breakable, listings}

\geometry{
  top=2.5cm,
  bottom=2.5cm,
  left=2.5cm,
  right=2.5cm
}

% ============================================================
%  COULEURS
% ============================================================
\definecolor{bleuPrimaire}{HTML}{1A3C6E}
\definecolor{bleuSecondaire}{HTML}{2E6DBD}
\definecolor{bleuClair}{HTML}{EAF2FB}
\definecolor{bleuAccent}{HTML}{4A90D9}
\definecolor{grisTexte}{HTML}{2C3E50}
\definecolor{grisClair}{HTML}{F4F6F9}
\definecolor{vertCode}{HTML}{27AE60}
\definecolor{orangeNote}{HTML}{E67E22}
\definecolor{rougeImportant}{HTML}{E74C3C}
\definecolor{violetSubtil}{HTML}{8E44AD}
\definecolor{codebg}{HTML}{1E2A3A}
\definecolor{codetext}{HTML}{ECF0F1}

% ============================================================
%  STYLE DE CODE
% ============================================================
\lstdefinestyle{pythonstyle}{
  language=Python,
  backgroundcolor=\color{codebg},
  basicstyle=\ttfamily\small\color{codetext},
  keywordstyle=\color{bleuAccent}\bfseries,
  stringstyle=\color{vertCode},
  commentstyle=\color{gray}\itshape,
  numberstyle=\tiny\color{gray},
  numbers=left,
  numbersep=8pt,
  stepnumber=1,
  showstringspaces=false,
  breaklines=true,
  frame=none,
  tabsize=4,
  captionpos=b,
  morekeywords={print, input, type, int, float, str, bool, True, False, and, or, not}
}

% ============================================================
%  EN-TÊTE ET PIED DE PAGE
% ============================================================
\pagestyle{fancy}
\fancyhf{}
\renewcommand{\headrulewidth}{0pt}
\fancyhead[L]{%
  \begin{tikzpicture}[remember picture, overlay]
    \fill[bleuPrimaire] (0,0) rectangle (\paperwidth, 1.1cm);
  \end{tikzpicture}%
  \color{white}\small\textbf{Cours Python — Mon Apprentissage}
}
\fancyhead[R]{\color{white}\small\textbf{Chapitre 4 : Entrées et Sorties}}
\fancyfoot[C]{%
  \begin{tikzpicture}[remember picture, overlay]
    \fill[bleuClair] (0,0) rectangle (\paperwidth, 0.7cm);
    \draw[bleuSecondaire, line width=1pt] (0, 0.7cm) -- (\paperwidth, 0.7cm);
  \end{tikzpicture}%
  \color{bleuPrimaire}\small\thepage
}

% ============================================================
%  STYLE DES TITRES
% ============================================================
\titleformat{\section}{%
  \color{bleuPrimaire}\Large\bfseries
}{}{0em}{}[\vspace{-0.3em}\textcolor{bleuSecondaire}{\rule{\linewidth}{1.5pt}}]

\titleformat{\subsection}{%
  \color{bleuSecondaire}\large\bfseries
}{}{0em}{}

\titlespacing{\section}{0pt}{1.5em}{0.8em}
\titlespacing{\subsection}{0pt}{1em}{0.5em}

% ============================================================
%  BOÎTES PERSONNALISÉES
% ============================================================
\newtcolorbox{boiteRetenir}{
  enhanced, breakable,
  colback=bleuClair, colframe=bleuSecondaire,
  arc=4pt, boxrule=1.5pt,
  left=10pt, right=10pt, top=8pt, bottom=8pt,
  title={\faLightbulb\quad À retenir},
  fonttitle=\bfseries\color{bleuPrimaire},
  attach boxed title to top left={yshift=-2mm, xshift=5mm},
  boxed title style={colback=bleuClair, colframe=bleuSecondaire}
}

\newtcolorbox{boiteNote}{
  enhanced, breakable,
  colback=orange!8, colframe=orangeNote,
  arc=4pt, boxrule=1.5pt,
  left=10pt, right=10pt, top=8pt, bottom=8pt,
  title={\faExclamationTriangle\quad Note importante},
  fonttitle=\bfseries\color{orangeNote},
  attach boxed title to top left={yshift=-2mm, xshift=5mm},
  boxed title style={colback=orange!8, colframe=orangeNote}
}

\newtcolorbox{boiteErreur}{
  enhanced, breakable,
  colback=red!5, colframe=rougeImportant,
  arc=4pt, boxrule=1.5pt,
  left=10pt, right=10pt, top=8pt, bottom=8pt,
  title={\faBug\quad Erreur fréquente},
  fonttitle=\bfseries\color{rougeImportant},
  attach boxed title to top left={yshift=-2mm, xshift=5mm},
  boxed title style={colback=red!5, colframe=rougeImportant}
}

% NOUVELLE BOÎTE — Difficultés et subtilités
\newtcolorbox{boiteDifficulte}{
  enhanced, breakable,
  colback=violetSubtil!8, colframe=violetSubtil,
  arc=4pt, boxrule=1.5pt,
  left=10pt, right=10pt, top=8pt, bottom=8pt,
  title={\faBrain\quad Ce qui m'a résisté},
  fonttitle=\bfseries\color{violetSubtil},
  attach boxed title to top left={yshift=-2mm, xshift=5mm},
  boxed title style={colback=violetSubtil!8, colframe=violetSubtil}
}

\newtcblisting{codeblock}[1]{
  enhanced, breakable,
  listing only,
  listing style=pythonstyle,
  colback=codebg, colframe=bleuSecondaire,
  arc=6pt, boxrule=1pt,
  title={\faCode\quad #1},
  fonttitle=\bfseries\color{white}\small,
  attach boxed title to top left={yshift=-2mm, xshift=5mm},
  boxed title style={colback=bleuSecondaire, colframe=bleuSecondaire},
  left=6pt, right=6pt, top=8pt, bottom=6pt
}

% ============================================================
%  PAGE DE TITRE
% ============================================================
\begin{document}

\begin{titlepage}
  \begin{tikzpicture}[remember picture, overlay]
    \fill[bleuPrimaire] (current page.north west) rectangle
      ([yshift=-10cm] current page.north east);
    \fill[bleuSecondaire] ([yshift=-10cm] current page.north west) rectangle
      ([yshift=-11cm] current page.north east);
    \fill[grisClair] ([yshift=-11cm] current page.north west) rectangle
      (current page.south east);
    \fill[bleuAccent, opacity=0.15]
      ([xshift=12cm, yshift=-5cm] current page.north west) circle (5cm);
    \fill[bleuAccent, opacity=0.08]
      ([xshift=15cm, yshift=-3cm] current page.north west) circle (3cm);
  \end{tikzpicture}

  \vspace*{2cm}

  \begin{center}
    {\fontsize{14}{16}\selectfont\color{bleuClair}
      \textbf{MON APPRENTISSAGE — SERIES PYTHON}}\\[0.5cm]
    {\fontsize{42}{48}\selectfont\color{white}\textbf{Python}}\\[0.2cm]
    {\fontsize{20}{24}\selectfont\color{bleuAccent}
      \textbf{Du zéro à l'expert}}\\[0.3cm]
    \textcolor{white!70}{\rule{8cm}{0.5pt}}\\[0.5cm]
    {\fontsize{16}{20}\selectfont\color{white}
      \textbf{Chapitre 4 — Les Entrées et Sorties}}\\[0.2cm]
    {\fontsize{13}{16}\selectfont\color{bleuClair}
      \textit{Faire parler Python — et lui faire écouter}}
  \end{center}

  \vspace{3.5cm}

  \begin{center}
    \begin{tikzpicture}
      \fill[bleuClair, rounded corners=8pt] (0,0) rectangle (12cm, 3.2cm);
      \draw[bleuSecondaire, line width=1.5pt, rounded corners=8pt]
        (0,0) rectangle (12cm, 3.2cm);
      \node[anchor=west] at (0.4, 2.5)
        {\color{bleuPrimaire}\small\textbf{Auteur :}};
      \node[anchor=west] at (0.4, 1.9)
        {\color{grisTexte}\large\textbf{KETOTSA AMÉVI CLAUDE}};
      \node[anchor=west] at (0.4, 1.2)
        {\color{bleuPrimaire}\small\textbf{Date de publication :}};
      \node[anchor=west] at (0.4, 0.6)
        {\color{grisTexte}\small\today};
    \end{tikzpicture}
  \end{center}

  \vfill
  \begin{center}
    \color{bleuSecondaire}\small\faLinkedin\quad
    \textit{Publié dans le cadre de mon parcours d'apprentissage Python}
  \end{center}
\end{titlepage}

% ============================================================
%  TABLE DES MATIÈRES
% ============================================================
\newpage
\tableofcontents
\newpage

% ============================================================
%  CHAPITRE 4
% ============================================================

\section*{\faPython\quad Chapitre 4 — Les Entrées et Sorties}
\addcontentsline{toc}{section}{Chapitre 4 — Les Entrées et Sorties}

\vspace{0.3cm}

\begin{boiteRetenir}
Un programme qui ne communique pas ne sert à rien. Dans ce chapitre, j'apprends comment Python \textbf{affiche des informations} à l'utilisateur avec \texttt{print()}, et comment il \textbf{reçoit des données} avec \texttt{input()}. C'est ici que les programmes commencent à véritablement \textit{vivre}.
\end{boiteRetenir}

\vspace{0.5cm}

% -----------------------------------------------------------
\subsection{La fonction print() — Afficher des données}

\texttt{print()} est la fonction la plus utilisée en Python. Elle affiche n'importe quelle valeur dans le terminal.

\begin{codeblock}{print() — Les bases}
# Afficher du texte simple
print("Bonjour tout le monde !")

# Afficher plusieurs valeurs separees par une virgule
print("Nom :", "Alice", "| Age :", 25)
# → Nom : Alice | Age : 25

# Afficher une variable
prenom = "Bob"
age = 30
print(prenom)
print(age)

# Afficher plusieurs variables ensemble
print(prenom, age)
# → Bob 30
\end{codeblock}

% -----------------------------------------------------------
\subsection{Les paramètres avancés de print()}

\texttt{print()} possède des paramètres cachés qui le rendent très puissant.

\begin{codeblock}{Paramètres sep et end}
# sep — definit le separateur entre les valeurs
# (par defaut : un espace)
print("Paris", "Lyon", "Marseille", sep=" | ")
# → Paris | Lyon | Marseille

print("02", "03", "2026", sep="-")
# → 02-03-2026

print("A", "B", "C", sep="")
# → ABC

# end — definit ce qui s'affiche a la fin
# (par defaut : saut de ligne \n)
print("Chargement", end="")
print("...")
# → Chargement...

print("Etape 1", end=" → ")
print("Etape 2", end=" → ")
print("Etape 3")
# → Etape 1 → Etape 2 → Etape 3
\end{codeblock}

\begin{boiteDifficulte}
Le paramètre \texttt{end} m'a surpris au début. Je pensais que \texttt{print()} ajoutait toujours un retour à la ligne — c'est son comportement \textbf{par défaut}, mais on peut totalement le contrôler. Comprendre que \texttt{print("texte")} est en réalité \texttt{print("texte", end="\textbackslash n")} a changé ma façon de voir la fonction.
\end{boiteDifficulte}

% -----------------------------------------------------------
\subsection{Le formatage des chaînes}

Python offre trois façons de formater du texte. La méthode moderne et recommandée est le \textbf{f-string}.

\begin{codeblock}{Les 3 méthodes de formatage}
prenom = "Alice"
age = 25
taille = 1.68

# Methode 1 — Concatenation (ancienne, deconseillee)
print("Je m'appelle " + prenom + " et j'ai " + str(age) + " ans.")

# Methode 2 — format() (intermediaire)
print("Je m'appelle {} et j'ai {} ans.".format(prenom, age))

# Methode 3 — f-string (moderne, recommandee ✅)
print(f"Je m'appelle {prenom} et j'ai {age} ans.")
# → Je m'appelle Alice et j'ai 25 ans.
\end{codeblock}

\begin{codeblock}{f-strings — Puissance et précision}
prix = 19.9999
pi = 3.14159265

# Arrondir les decimales dans un f-string
print(f"Prix : {prix:.2f} €")     # → Prix : 20.00 €
print(f"Pi ≈ {pi:.4f}")           # → Pi ≈ 3.1416

# Aligner du texte sur une largeur fixe
print(f"{'Produit':<15} {'Prix':>8}")
print(f"{'Café':<15} {2.50:>8.2f}")
print(f"{'Croissant':<15} {1.20:>8.2f}")
# →
# Produit              Prix
# Cafe                 2.50
# Croissant            1.20

# Expressions dans les f-strings
a = 10
b = 3
print(f"{a} x {b} = {a * b}")   # → 10 x 3 = 30
print(f"Majeur : {a >= 18}")    # → Majeur : False
\end{codeblock}

\begin{boiteDifficulte}
La syntaxe \texttt{:\,.2f} dans les f-strings m'a demandé du temps. Le \texttt{.2f} signifie : \textit{"affiche ce nombre flottant avec exactement 2 décimales"}. Le \texttt{<} et \texttt{>} pour l'alignement m'ont aussi surpris — mais une fois compris, c'est redoutablement pratique pour créer des tableaux dans le terminal.
\end{boiteDifficulte}

% -----------------------------------------------------------
\subsection{Les caractères spéciaux dans les chaînes}

Python utilise des \textbf{séquences d'échappement} pour représenter des caractères spéciaux.

\begin{center}
\begin{tikzpicture}
  \foreach \car/\desc/\y in {
    {\textbackslash n}/{Saut de ligne}/5.5,
    {\textbackslash t}/{Tabulation horizontale}/4.5,
    {\textbackslash r}/{Retour chariot}/3.5,
    {\textbackslash\textbackslash}/{Antislash littéral}/2.5,
    {\textbackslash "}/{Guillemet double dans une chaîne}/1.5,
    {\textbackslash '}/{Guillemet simple dans une chaîne}/0.5
  }{
    \fill[bleuClair] (0, \y-0.4) rectangle (13.5, \y+0.4);
    \draw[bleuSecondaire!30] (0, \y-0.4) -- (13.5, \y-0.4);
    \node[font=\ttfamily\bfseries\small, text=bleuSecondaire,
      anchor=west] at (0.3, \y) {\car};
    \node[font=\small, text=grisTexte, anchor=west] at (4.0, \y) {\desc};
  }
  \draw[bleuSecondaire, line width=1pt] (0, 0.1) rectangle (13.5, 6.0);
\end{tikzpicture}
\end{center}

\begin{codeblock}{Caractères spéciaux en action}
# Saut de ligne \n
print("Ligne 1\nLigne 2\nLigne 3")

# Tabulation \t
print("Nom\tAge\tVille")
print("Alice\t25\tParis")
print("Bob\t30\tLyon")

# Chemin de fichier Windows — attention au \
# ❌ Mauvais :  print("C:\nouveau\fichier.txt")
# ✅ Solution 1 : doubler les antislashs
print("C:\\nouveau\\fichier.txt")
# ✅ Solution 2 : raw string avec r""
print(r"C:\nouveau\fichier.txt")
\end{codeblock}

% -----------------------------------------------------------
\subsection{La fonction input() — Recevoir des données}

\texttt{input()} permet au programme de \textbf{demander une information} à l'utilisateur. Il s'arrête et attend que l'utilisateur tape quelque chose et appuie sur Entrée.

\begin{codeblock}{input() — Les bases}
# Demander une information
prenom = input("Quel est ton prenom ? ")
print(f"Bonjour, {prenom} !")

# input() retourne TOUJOURS une chaine (str)
age_texte = input("Quel est ton age ? ")
print(type(age_texte))   # <class 'str'>

# Convertir pour faire des calculs
age = int(input("Quel est ton age ? "))
annee_naissance = 2026 - age
print(f"Tu es ne(e) en {annee_naissance}.")
\end{codeblock}

\begin{codeblock}{Programme interactif complet}
# Mini programme de presentation
print("=" * 40)
print("   GENERATEUR DE PRESENTATION")
print("=" * 40)

prenom = input("\nTon prenom : ")
age    = int(input("Ton age : "))
ville  = input("Ta ville : ")
metier = input("Ton metier : ")

print("\n" + "=" * 40)
print(f"  Bonjour, je m'appelle {prenom}.")
print(f"  J'ai {age} ans et j'habite a {ville}.")
print(f"  Je travaille comme {metier}.")
print(f"  Dans {65 - age} ans, je serai a la retraite !")
print("=" * 40)
\end{codeblock}

\begin{boiteDifficulte}
La subtilité qui m'a le plus piégé : \texttt{input()} retourne \textbf{toujours} une chaîne de caractères — même si l'utilisateur tape \texttt{25}. J'ai eu une erreur en essayant d'additionner directement le résultat d'un \texttt{input()} avec un nombre. Python ne convertit pas automatiquement. Il faut \textbf{toujours} encadrer avec \texttt{int()} ou \texttt{float()} si on veut faire des calculs. Cette règle semble simple, mais elle m'a coûté plusieurs minutes de débogage avant de comprendre.
\end{boiteDifficulte}

% -----------------------------------------------------------
\subsection{Gérer les erreurs de saisie}

Que se passe-t-il si l'utilisateur tape du texte alors qu'on attend un nombre ?

\begin{codeblock}{Erreur classique et protection}
# ❌ Si l'utilisateur tape "abc" au lieu d'un nombre
# int(input(...)) va lever une erreur : ValueError

# ✅ Protection avec try/except (apercu — chapitre 19)
try:
    age = int(input("Ton age : "))
    print(f"Dans 10 ans, tu auras {age + 10} ans.")
except ValueError:
    print("Erreur : veuillez entrer un nombre entier.")

# ✅ Validation simple avec isdigit()
saisie = input("Entrez un nombre : ")
if saisie.isdigit():
    nombre = int(saisie)
    print(f"Votre nombre au carre : {nombre ** 2}")
else:
    print("Ce n'est pas un nombre valide.")
\end{codeblock}

\begin{boiteNote}
La gestion complète des erreurs sera couverte au \textbf{Chapitre 19}. Mais il est important de savoir dès maintenant que \texttt{int(input())} peut \textit{planter} si l'utilisateur ne coopère pas. Un bon programme anticipe toujours les erreurs humaines.
\end{boiteNote}

% -----------------------------------------------------------
\subsection{Ce qui m'a le plus surpris dans ce chapitre}

\begin{boiteDifficulte}
\textbf{Trois subtilités qui m'ont marqué :}

\vspace{0.5em}
\textbf{1. La multiplication de chaînes.}
En découvrant \texttt{print("=" * 40)}, j'ai réalisé qu'on pouvait multiplier une chaîne par un nombre en Python. C'est élégant et inattendu — \texttt{"=" * 40} produit une ligne de 40 signes égal. Aucun autre langage que je connaisse ne fait ça aussi naturellement.

\vspace{0.5em}
\textbf{2. Les f-strings sont des expressions, pas juste du texte.}
On peut écrire \texttt{f"\{2 + 2\}"} et obtenir \texttt{4}. On peut appeler des fonctions, faire des calculs, tout ça à l'intérieur des accolades. Ce n'est pas juste du formatage — c'est du Python à part entière, intégré dans du texte.

\vspace{0.5em}
\textbf{3. Le \texttt{r""} pour les raw strings.}
Écrire un chemin Windows comme \texttt{C:\textbackslash Users\textbackslash fichier} provoque des erreurs à cause des antislashs. La solution \texttt{r"C:\textbackslash Users\textbackslash fichier"} désactive l'interprétation des séquences d'échappement. Un détail technique qui peut faire perdre beaucoup de temps si on ne le connaît pas.
\end{boiteDifficulte}

% -----------------------------------------------------------
\subsection{Résumé du Chapitre 4}

\begin{tcolorbox}[
  enhanced, breakable,
  colback=bleuPrimaire,
  colframe=bleuPrimaire,
  arc=8pt,
  left=15pt, right=15pt, top=10pt, bottom=10pt
]
  \color{white}
  \textbf{\large\faClipboardList\quad Ce que j'ai appris dans ce chapitre :}

  \vspace{0.5em}
  \begin{itemize}[leftmargin=1.5em, itemsep=6pt]
    \item[\textcolor{bleuAccent}{\faCheckCircle}]
      \texttt{print()} affiche n'importe quelle valeur — avec \texttt{sep} et \texttt{end} pour contrôler la mise en forme.
    \item[\textcolor{bleuAccent}{\faCheckCircle}]
      Les \textbf{f-strings} sont la méthode moderne pour formater du texte — puissantes et lisibles.
    \item[\textcolor{bleuAccent}{\faCheckCircle}]
      \texttt{:\,.2f} dans un f-string contrôle la précision des décimales.
    \item[\textcolor{bleuAccent}{\faCheckCircle}]
      Les \textbf{séquences d'échappement} : \texttt{\textbackslash n}, \texttt{\textbackslash t}, \texttt{r""} pour les raw strings.
    \item[\textcolor{bleuAccent}{\faCheckCircle}]
      \texttt{input()} reçoit toujours une \texttt{str} — toujours convertir avec \texttt{int()} ou \texttt{float()}.
    \item[\textcolor{bleuAccent}{\faCheckCircle}]
      Anticiper les erreurs de saisie avec \texttt{isdigit()} ou \texttt{try/except}.
    \item[\textcolor{bleuAccent}{\faCheckCircle}]
      \texttt{"=" * 40} — multiplier une chaîne par un entier est une des élégances uniques de Python.
  \end{itemize}

  \vspace{0.5em}
  \textit{\textcolor{bleuClair}{"Un programme qui ne communique pas avec l'humain
  n'est qu'une boîte noire. L'entrée et la sortie, c'est le dialogue."}}
\end{tcolorbox}

% -----------------------------------------------------------
\vspace{1cm}
\begin{center}
  \begin{tikzpicture}
    \fill[bleuClair, rounded corners=6pt] (0,0) rectangle (16, 1.5);
    \draw[bleuSecondaire, line width=1pt, rounded corners=6pt]
      (0,0) rectangle (16, 1.5);
    \node[font=\small, text=bleuPrimaire] at (8, 0.75) {%
      \faLinkedin\quad \textbf{KETOTSA AMÉVI CLAUDE} — Mon apprentissage Python
      \quad|\quad Chapitre 4 sur 34
    };
  \end{tikzpicture}
\end{center}

\end{document}
