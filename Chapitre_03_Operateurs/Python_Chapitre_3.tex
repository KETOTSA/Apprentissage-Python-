
\documentclass[12pt, a4paper]{article}

% ============================================================
%  PACKAGES
% ============================================================
\usepackage[utf8]{inputenc}
\usepackage[T1]{fontenc}
\usepackage[french]{babel}
\usepackage{geometry}
\usepackage{graphicx}
\usepackage{xcolor}
\usepackage{tikz}
\usepackage{mdframed}
\usepackage{listings}
\usepackage{fontawesome5}
\usepackage{fancyhdr}
\usepackage{titlesec}
\usepackage{enumitem}
\usepackage{lmodern}
\usepackage{microtype}
\usepackage{hyperref}
\usepackage{tcolorbox}
\tcbuselibrary{skins, breakable, listings}

\geometry{
  top=2.5cm,
  bottom=2.5cm,
  left=2.5cm,
  right=2.5cm
}

% ============================================================
%  COULEURS
% ============================================================
\definecolor{bleuPrimaire}{HTML}{1A3C6E}
\definecolor{bleuSecondaire}{HTML}{2E6DBD}
\definecolor{bleuClair}{HTML}{EAF2FB}
\definecolor{bleuAccent}{HTML}{4A90D9}
\definecolor{grisTexte}{HTML}{2C3E50}
\definecolor{grisClair}{HTML}{F4F6F9}
\definecolor{vertCode}{HTML}{27AE60}
\definecolor{orangeNote}{HTML}{E67E22}
\definecolor{rougeImportant}{HTML}{E74C3C}
\definecolor{codebg}{HTML}{1E2A3A}
\definecolor{codetext}{HTML}{ECF0F1}

% ============================================================
%  STYLE DE CODE
% ============================================================
\lstdefinestyle{pythonstyle}{
  language=Python,
  backgroundcolor=\color{codebg},
  basicstyle=\ttfamily\small\color{codetext},
  keywordstyle=\color{bleuAccent}\bfseries,
  stringstyle=\color{vertCode},
  commentstyle=\color{gray}\itshape,
  numberstyle=\tiny\color{gray},
  numbers=left,
  numbersep=8pt,
  stepnumber=1,
  showstringspaces=false,
  breaklines=true,
  frame=none,
  tabsize=4,
  captionpos=b,
  morekeywords={print, input, type, int, float, str, bool, True, False, and, or, not}
}

% ============================================================
%  EN-TÊTE ET PIED DE PAGE
% ============================================================
\pagestyle{fancy}
\fancyhf{}
\renewcommand{\headrulewidth}{0pt}
\fancyhead[L]{%
  \begin{tikzpicture}[remember picture, overlay]
    \fill[bleuPrimaire] (0,0) rectangle (\paperwidth, 1.1cm);
  \end{tikzpicture}%
  \color{white}\small\textbf{Cours Python — Mon Apprentissage}
}
\fancyhead[R]{\color{white}\small\textbf{Chapitre 3 : Les Opérateurs}}
\fancyfoot[C]{%
  \begin{tikzpicture}[remember picture, overlay]
    \fill[bleuClair] (0,0) rectangle (\paperwidth, 0.7cm);
    \draw[bleuSecondaire, line width=1pt] (0, 0.7cm) -- (\paperwidth, 0.7cm);
  \end{tikzpicture}%
  \color{bleuPrimaire}\small\thepage
}

% ============================================================
%  STYLE DES TITRES
% ============================================================
\titleformat{\section}{%
  \color{bleuPrimaire}\Large\bfseries
}{}{0em}{}[\vspace{-0.3em}\textcolor{bleuSecondaire}{\rule{\linewidth}{1.5pt}}]

\titleformat{\subsection}{%
  \color{bleuSecondaire}\large\bfseries
}{}{0em}{}

\titlespacing{\section}{0pt}{1.5em}{0.8em}
\titlespacing{\subsection}{0pt}{1em}{0.5em}

% ============================================================
%  BOÎTES PERSONNALISÉES
% ============================================================
\newtcolorbox{boiteRetenir}{
  enhanced, breakable,
  colback=bleuClair, colframe=bleuSecondaire,
  arc=4pt, boxrule=1.5pt,
  left=10pt, right=10pt, top=8pt, bottom=8pt,
  title={\faLightbulb\quad À retenir},
  fonttitle=\bfseries\color{bleuPrimaire},
  attach boxed title to top left={yshift=-2mm, xshift=5mm},
  boxed title style={colback=bleuClair, colframe=bleuSecondaire}
}

\newtcolorbox{boiteNote}{
  enhanced, breakable,
  colback=orange!8, colframe=orangeNote,
  arc=4pt, boxrule=1.5pt,
  left=10pt, right=10pt, top=8pt, bottom=8pt,
  title={\faExclamationTriangle\quad Note importante},
  fonttitle=\bfseries\color{orangeNote},
  attach boxed title to top left={yshift=-2mm, xshift=5mm},
  boxed title style={colback=orange!8, colframe=orangeNote}
}

\newtcolorbox{boiteErreur}{
  enhanced, breakable,
  colback=red!5, colframe=rougeImportant,
  arc=4pt, boxrule=1.5pt,
  left=10pt, right=10pt, top=8pt, bottom=8pt,
  title={\faBug\quad Erreur fréquente},
  fonttitle=\bfseries\color{rougeImportant},
  attach boxed title to top left={yshift=-2mm, xshift=5mm},
  boxed title style={colback=red!5, colframe=rougeImportant}
}

\newtcblisting{codeblock}[1]{
  enhanced, breakable,
  listing only,
  listing style=pythonstyle,
  colback=codebg, colframe=bleuSecondaire,
  arc=6pt, boxrule=1pt,
  title={\faCode\quad #1},
  fonttitle=\bfseries\color{white}\small,
  attach boxed title to top left={yshift=-2mm, xshift=5mm},
  boxed title style={colback=bleuSecondaire, colframe=bleuSecondaire},
  left=6pt, right=6pt, top=8pt, bottom=6pt
}

% ============================================================
%  PAGE DE TITRE
% ============================================================
\begin{document}

\begin{titlepage}
  \begin{tikzpicture}[remember picture, overlay]
    \fill[bleuPrimaire] (current page.north west) rectangle
      ([yshift=-10cm] current page.north east);
    \fill[bleuSecondaire] ([yshift=-10cm] current page.north west) rectangle
      ([yshift=-11cm] current page.north east);
    \fill[grisClair] ([yshift=-11cm] current page.north west) rectangle
      (current page.south east);
    \fill[bleuAccent, opacity=0.15]
      ([xshift=12cm, yshift=-5cm] current page.north west) circle (5cm);
    \fill[bleuAccent, opacity=0.08]
      ([xshift=15cm, yshift=-3cm] current page.north west) circle (3cm);
  \end{tikzpicture}

  \vspace*{2cm}

  \begin{center}
    {\fontsize{14}{16}\selectfont\color{bleuClair}
      \textbf{MON APPRENTISSAGE — SERIES PYTHON}}\\[0.5cm]
    {\fontsize{42}{48}\selectfont\color{white}\textbf{Python}}\\[0.2cm]
    {\fontsize{20}{24}\selectfont\color{bleuAccent}
      \textbf{Du zéro à l'expert}}\\[0.3cm]
    \textcolor{white!70}{\rule{8cm}{0.5pt}}\\[0.5cm]
    {\fontsize{16}{20}\selectfont\color{white}
      \textbf{Chapitre 3 — Les Opérateurs}}\\[0.2cm]
    {\fontsize{13}{16}\selectfont\color{bleuClair}
      \textit{Les outils qui donnent vie aux calculs et à la logique}}
  \end{center}

  \vspace{3.5cm}

  \begin{center}
    \begin{tikzpicture}
      \fill[bleuClair, rounded corners=8pt] (0,0) rectangle (12cm, 3.2cm);
      \draw[bleuSecondaire, line width=1.5pt, rounded corners=8pt]
        (0,0) rectangle (12cm, 3.2cm);
      \node[anchor=west] at (0.4, 2.5)
        {\color{bleuPrimaire}\small\textbf{Auteur :}};
      \node[anchor=west] at (0.4, 1.9)
        {\color{grisTexte}\large\textbf{KETOTSA AMÉVI CLAUDE}};
      \node[anchor=west] at (0.4, 1.2)
        {\color{bleuPrimaire}\small\textbf{Date de publication :}};
      \node[anchor=west] at (0.4, 0.6)
        {\color{grisTexte}\small\today};
    \end{tikzpicture}
  \end{center}

  \vfill
  \begin{center}
    \color{bleuSecondaire}\small\faLinkedin\quad
    \textit{Publié dans le cadre de mon parcours d'apprentissage Python}
  \end{center}
\end{titlepage}

% ============================================================
%  TABLE DES MATIÈRES
% ============================================================
\newpage
\tableofcontents
\newpage

% ============================================================
%  CHAPITRE 3
% ============================================================

\section*{\faPython\quad Chapitre 3 — Les Opérateurs}
\addcontentsline{toc}{section}{Chapitre 3 — Les Opérateurs}

\vspace{0.3cm}

\begin{boiteRetenir}
Les opérateurs sont les \textbf{outils fondamentaux} de Python. Ils permettent d'effectuer des calculs, de comparer des valeurs, de combiner des conditions et de modifier des variables. Sans eux, un programme ne peut rien décider, rien calculer, rien produire.
\end{boiteRetenir}

\vspace{0.5cm}

% -----------------------------------------------------------
\subsection{Les opérateurs arithmétiques}

Les opérateurs arithmétiques permettent d'effectuer des \textbf{calculs mathématiques} de base.

\begin{center}
\begin{tikzpicture}
  \foreach \op/\nom/\exemple/\pos in {
    {+}/{Addition}/{5 + 3 = 8}/0,
    {-}/{Soustraction}/{5 - 3 = 2}/3.2,
    {*}/{Multiplication}/{5 * 3 = 15}/6.4,
    {/}/{Division}/{5 / 2 = 2.5}/9.6
  }{
    \fill[bleuClair, rounded corners=5pt] (\pos, 0) rectangle (\pos+2.8, 2.5);
    \draw[bleuSecondaire, line width=1pt, rounded corners=5pt]
      (\pos, 0) rectangle (\pos+2.8, 2.5);
    \node[font=\bfseries\Large, text=bleuPrimaire] at (\pos+1.4, 2.0) {\op};
    \node[font=\small\bfseries, text=bleuSecondaire] at (\pos+1.4, 1.4) {\nom};
    \node[font=\ttfamily\tiny, text=grisTexte] at (\pos+1.4, 0.7) {\exemple};
  }
\end{tikzpicture}
\end{center}

\vspace{0.4cm}

\begin{center}
\begin{tikzpicture}
  \foreach \op/\nom/\exemple/\pos in {
    {//}/{Division entière}/{5 // 2 = 2}/0,
    {\%}/{Modulo (reste)}/{5 \% 2 = 1}/3.2,
    {**}/{Puissance}/{2 ** 8 = 256}/6.4
  }{
    \fill[bleuPrimaire!10, rounded corners=5pt] (\pos, 0) rectangle (\pos+2.8, 2.5);
    \draw[bleuPrimaire, line width=1pt, rounded corners=5pt]
      (\pos, 0) rectangle (\pos+2.8, 2.5);
    \node[font=\bfseries\Large, text=bleuPrimaire] at (\pos+1.4, 2.0) {\op};
    \node[font=\small\bfseries, text=bleuSecondaire] at (\pos+1.4, 1.4) {\nom};
    \node[font=\ttfamily\tiny, text=grisTexte] at (\pos+1.4, 0.7) {\exemple};
  }
\end{tikzpicture}
\end{center}

\begin{codeblock}{Opérateurs arithmétiques en action}
a = 17
b = 5

print(a + b)    # Addition         → 22
print(a - b)    # Soustraction     → 12
print(a * b)    # Multiplication   → 85
print(a / b)    # Division reelle  → 3.4
print(a // b)   # Division entiere → 3
print(a % b)    # Modulo (reste)   → 2
print(a ** b)   # Puissance        → 1419857

# Exemple concret : calculer le prix TTC
prix_ht = 200
tva = 0.20
prix_ttc = prix_ht * (1 + tva)
print(f"Prix TTC : {prix_ttc} €")   # Prix TTC : 240.0 €
\end{codeblock}

\begin{boiteRetenir}
Le \textbf{modulo} \texttt{\%} est très utile en programmation : il permet de savoir si un nombre est pair ou impair, de créer des cycles, ou de limiter des valeurs dans un intervalle.
\texttt{nombre \% 2 == 0} → le nombre est pair.
\end{boiteRetenir}

% -----------------------------------------------------------
\subsection{Les opérateurs de comparaison}

Les opérateurs de comparaison \textbf{comparent deux valeurs} et retournent toujours un booléen : \texttt{True} ou \texttt{False}.

\begin{codeblock}{Opérateurs de comparaison}
a = 10
b = 20

print(a == b)   # Egal a           → False
print(a != b)   # Different de     → True
print(a < b)    # Inferieur a      → True
print(a > b)    # Superieur a      → False
print(a <= b)   # Inferieur ou egal → True
print(a >= b)   # Superieur ou egal → False

# Comparer des chaines
prenom1 = "Alice"
prenom2 = "Bob"
print(prenom1 == prenom2)   # False
print(prenom1 != prenom2)   # True

# Exemple concret : verifier l'age
age = 17
est_majeur = age >= 18
print(est_majeur)   # False
\end{codeblock}

\begin{boiteErreur}
Ne confondez jamais \texttt{=} et \texttt{==} !
\texttt{=} est une \textbf{assignation} (on donne une valeur à une variable).
\texttt{==} est une \textbf{comparaison} (on vérifie si deux valeurs sont égales).
C'est l'une des erreurs les plus fréquentes chez les débutants.
\end{boiteErreur}

% -----------------------------------------------------------
\subsection{Les opérateurs logiques}

Les opérateurs logiques permettent de \textbf{combiner plusieurs conditions} ensemble.

\begin{center}
\begin{tikzpicture}
  % and
  \fill[bleuClair, rounded corners=6pt] (0, 0) rectangle (4.2, 4);
  \draw[bleuSecondaire, line width=1.5pt, rounded corners=6pt]
    (0,0) rectangle (4.2, 4);
  \node[font=\bfseries, text=bleuPrimaire] at (2.1, 3.6) {\texttt{and}};
  \node[font=\small, text=grisTexte, align=center] at (2.1, 2.9)
    {Les DEUX conditions};
  \node[font=\small, text=grisTexte, align=center] at (2.1, 2.4)
    {doivent être \textbf{True}};
  \draw[bleuSecondaire!40, dashed] (0.2, 2.0) -- (4.0, 2.0);
  \node[font=\ttfamily\tiny, text=vertCode] at (2.1, 1.5)
    {True and True → True};
  \node[font=\ttfamily\tiny, text=rougeImportant] at (2.1, 1.0)
    {True and False → False};
  \node[font=\ttfamily\tiny, text=rougeImportant] at (2.1, 0.5)
    {False and False → False};

  % or
  \fill[bleuClair, rounded corners=6pt] (4.6, 0) rectangle (8.8, 4);
  \draw[bleuSecondaire, line width=1.5pt, rounded corners=6pt]
    (4.6,0) rectangle (8.8, 4);
  \node[font=\bfseries, text=bleuPrimaire] at (6.7, 3.6) {\texttt{or}};
  \node[font=\small, text=grisTexte, align=center] at (6.7, 2.9)
    {AU MOINS UNE condition};
  \node[font=\small, text=grisTexte, align=center] at (6.7, 2.4)
    {doit être \textbf{True}};
  \draw[bleuSecondaire!40, dashed] (4.8, 2.0) -- (8.6, 2.0);
  \node[font=\ttfamily\tiny, text=vertCode] at (6.7, 1.5)
    {True or False → True};
  \node[font=\ttfamily\tiny, text=vertCode] at (6.7, 1.0)
    {False or True → True};
  \node[font=\ttfamily\tiny, text=rougeImportant] at (6.7, 0.5)
    {False or False → False};

  % not
  \fill[bleuClair, rounded corners=6pt] (9.2, 0) rectangle (13.4, 4);
  \draw[bleuSecondaire, line width=1.5pt, rounded corners=6pt]
    (9.2,0) rectangle (13.4, 4);
  \node[font=\bfseries, text=bleuPrimaire] at (11.3, 3.6) {\texttt{not}};
  \node[font=\small, text=grisTexte, align=center] at (11.3, 2.9)
    {INVERSE la valeur};
  \node[font=\small, text=grisTexte, align=center] at (11.3, 2.4)
    {du booléen};
  \draw[bleuSecondaire!40, dashed] (9.4, 2.0) -- (13.2, 2.0);
  \node[font=\ttfamily\tiny, text=rougeImportant] at (11.3, 1.5)
    {not True → False};
  \node[font=\ttfamily\tiny, text=vertCode] at (11.3, 1.0)
    {not False → True};
\end{tikzpicture}
\end{center}

\begin{codeblock}{Opérateurs logiques en action}
age = 20
a_un_billet = True
est_vip = False

# and — les deux conditions doivent etre vraies
peut_entrer = age >= 18 and a_un_billet
print(peut_entrer)   # True

# or — au moins une condition doit etre vraie
acces_special = est_vip or age >= 21
print(acces_special)   # False

# not — inverse la condition
porte_fermee = False
print(not porte_fermee)   # True (la porte est ouverte)

# Combinaison de plusieurs operateurs
a = 15
resultat = (a > 10) and (a < 20) and (a % 2 != 0)
print(resultat)   # True (15 est entre 10 et 20, et impair)
\end{codeblock}

% -----------------------------------------------------------
\subsection{Les opérateurs d'affectation}

Les opérateurs d'affectation permettent de \textbf{modifier une variable} de façon concise.

\begin{codeblock}{Opérateurs d'affectation}
score = 100

score += 10    # score = score + 10   → 110
print(score)

score -= 20    # score = score - 20   → 90
print(score)

score *= 2     # score = score * 2    → 180
print(score)

score //= 3    # score = score // 3   → 60
print(score)

score **= 2    # score = score ** 2   → 3600
print(score)

score %= 100   # score = score % 100  → 0
print(score)

# Exemple concret : compter des points dans un jeu
points = 0
points += 50    # Ennemi vaincu
points += 100   # Niveau termine
points -= 10    # Vie perdue
print(f"Score final : {points}")   # Score final : 140
\end{codeblock}

% -----------------------------------------------------------
\subsection{Les opérateurs d'identité et d'appartenance}

Python possède deux opérateurs spéciaux très utiles que l'on ne trouve pas dans tous les langages.

\begin{codeblock}{is, is not, in, not in}
# is / is not — verifie si deux variables pointent
# vers le MEME objet en memoire
a = None
print(a is None)       # True
print(a is not None)   # False

# in / not in — verifie si une valeur est dans
# une sequence (liste, chaine, etc.)
fruits = ["pomme", "banane", "cerise"]
print("banane" in fruits)       # True
print("mangue" not in fruits)   # True

# Avec les chaines
message = "Bonjour tout le monde"
print("monde" in message)    # True
print("Python" in message)   # False

# Exemple concret : vérifier un acces
admins = ["Alice", "Bob", "Charlie"]
utilisateur = "Alice"
if utilisateur in admins:
    print(f"{utilisateur} a les droits admin.")
\end{codeblock}

\begin{boiteRetenir}
\texttt{in} est l'un des opérateurs les plus élégants de Python. Il permet de vérifier l'appartenance à une collection en une seule expression, là où d'autres langages nécessiteraient une boucle entière.
\end{boiteRetenir}

% -----------------------------------------------------------
\subsection{La priorité des opérateurs}

Comme en mathématiques, Python applique une \textbf{priorité} entre les opérateurs. De la plus haute à la plus basse :

\begin{center}
\begin{tikzpicture}
  \foreach \prio/\op/\desc/\y in {
    {1 — Priorité max}/{\texttt{**}}/{Puissance}/6.5,
    {2}/{\texttt{+x, -x, \textasciitilde x}}/{Unaires}/5.5,
    {3}/{\texttt{*, /, //, \%}}/{Multiplication, Division}/4.5,
    {4}/{\texttt{+, -}}/{Addition, Soustraction}/3.5,
    {5}/{\texttt{<, >, <=, >=, ==, !=}}/{Comparaisons}/2.5,
    {6}/{\texttt{not}}/{NON logique}/1.5,
    {7 — Priorité min}/{\texttt{and, or}}/{ET / OU logique}/0.5
  }{
    \fill[bleuClair] (0, \y-0.4) rectangle (13.5, \y+0.4);
    \draw[bleuSecondaire!30] (0, \y-0.4) -- (13.5, \y-0.4);
    \node[font=\small\bfseries, text=bleuPrimaire, anchor=west] at (0.2, \y)
      {\prio};
    \node[font=\ttfamily\small, text=bleuSecondaire] at (5.5, \y) {\op};
    \node[font=\small, text=grisTexte, anchor=west] at (8.5, \y) {\desc};
  }
  \draw[bleuSecondaire, line width=1pt] (0, 0.1) rectangle (13.5, 7.0);
\end{tikzpicture}
\end{center}

\begin{codeblock}{Priorité des opérateurs — exemples}
# Sans parentheses — Python suit les priorites
resultat = 2 + 3 * 4
print(resultat)   # 14  (pas 20 — * avant +)

# Avec parentheses — vous controlez l'ordre
resultat = (2 + 3) * 4
print(resultat)   # 20

# Exemple complexe
x = 10
y = 3
z = 2
resultat = x + y ** z * 2 - 1
# Ordre : y**z=9, 9*2=18, 10+18=28, 28-1=27
print(resultat)   # 27

# Conseil : toujours utiliser des parentheses
# pour rendre le code lisible et eviter les erreurs
resultat_clair = x + ((y ** z) * 2) - 1
print(resultat_clair)   # 27
\end{codeblock}

\begin{boiteNote}
En cas de doute sur la priorité, utilisez toujours des \textbf{parenthèses}. Elles rendent le code plus lisible et évitent les erreurs de calcul inattendues.
\end{boiteNote}

% -----------------------------------------------------------
\subsection{Résumé du Chapitre 3}

\begin{tcolorbox}[
  enhanced, breakable,
  colback=bleuPrimaire,
  colframe=bleuPrimaire,
  arc=8pt,
  left=15pt, right=15pt, top=10pt, bottom=10pt
]
  \color{white}
  \textbf{\large\faClipboardList\quad Ce que j'ai appris dans ce chapitre :}

  \vspace{0.5em}
  \begin{itemize}[leftmargin=1.5em, itemsep=6pt]
    \item[\textcolor{bleuAccent}{\faCheckCircle}]
      Les \textbf{opérateurs arithmétiques} : \texttt{+, -, *, /, //, \%, **}
    \item[\textcolor{bleuAccent}{\faCheckCircle}]
      Les \textbf{opérateurs de comparaison} : \texttt{==, !=, <, >, <=, >=}
    \item[\textcolor{bleuAccent}{\faCheckCircle}]
      Les \textbf{opérateurs logiques} : \texttt{and, or, not}
    \item[\textcolor{bleuAccent}{\faCheckCircle}]
      Les \textbf{opérateurs d'affectation} : \texttt{+=, -=, *=, /=} \ldots
    \item[\textcolor{bleuAccent}{\faCheckCircle}]
      Les \textbf{opérateurs spéciaux} : \texttt{is, is not, in, not in}
    \item[\textcolor{bleuAccent}{\faCheckCircle}]
      La \textbf{priorité des opérateurs} — et pourquoi les parenthèses sont vos meilleures alliées.
    \item[\textcolor{bleuAccent}{\faCheckCircle}]
      Ne jamais confondre \texttt{=} (assignation) et \texttt{==} (comparaison).
  \end{itemize}

  \vspace{0.5em}
  \textit{\textcolor{bleuClair}{"La logique vous mènera d'un point A à un point B.
  L'imagination vous mènera partout." — Albert Einstein}}
\end{tcolorbox}

% -----------------------------------------------------------
\vspace{1cm}
\begin{center}
  \begin{tikzpicture}
    \fill[bleuClair, rounded corners=6pt] (0,0) rectangle (16, 1.5);
    \draw[bleuSecondaire, line width=1pt, rounded corners=6pt]
      (0,0) rectangle (16, 1.5);
    \node[font=\small, text=bleuPrimaire] at (8, 0.75) {%
      \faLinkedin\quad \textbf{KETOTSA AMÉVI CLAUDE} — Mon apprentissage Python
      \quad|\quad Chapitre 3 sur 34
    };
  \end{tikzpicture}
\end{center}

\end{document}






