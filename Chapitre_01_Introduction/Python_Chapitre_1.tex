\documentclass[12pt, a4paper]{article}

% ============================================================
%  PACKAGES
% ============================================================
\usepackage[utf8]{inputenc}
\usepackage[T1]{fontenc}
\usepackage[french]{babel}
\usepackage{geometry}
\usepackage{graphicx}
\usepackage{xcolor}
\usepackage{tikz}
\usepackage{mdframed}
\usepackage{listings}
\usepackage{fontawesome5}
\usepackage{fancyhdr}
\usepackage{titlesec}
\usepackage{enumitem}
\usepackage{lmodern}
\usepackage{microtype}
\usepackage{hyperref}
\usepackage{tcolorbox}
\tcbuselibrary{skins, breakable, listings}

\geometry{
  top=2.5cm,
  bottom=2.5cm,
  left=2.5cm,
  right=2.5cm
}

% ============================================================
%  COULEURS
% ============================================================
\definecolor{bleuPrimaire}{HTML}{1A3C6E}
\definecolor{bleuSecondaire}{HTML}{2E6DBD}
\definecolor{bleuClair}{HTML}{EAF2FB}
\definecolor{bleuAccent}{HTML}{4A90D9}
\definecolor{grisTexte}{HTML}{2C3E50}
\definecolor{grisClair}{HTML}{F4F6F9}
\definecolor{vertCode}{HTML}{27AE60}
\definecolor{orangeNote}{HTML}{E67E22}
\definecolor{rougeImportant}{HTML}{E74C3C}
\definecolor{codebg}{HTML}{1E2A3A}
\definecolor{codetext}{HTML}{ECF0F1}

% ============================================================
%  STYLE DE CODE
% ============================================================
\lstdefinestyle{pythonstyle}{
  language=Python,
  backgroundcolor=\color{codebg},
  basicstyle=\ttfamily\small\color{codetext},
  keywordstyle=\color{bleuAccent}\bfseries,
  stringstyle=\color{vertCode},
  commentstyle=\color{gray}\itshape,
  numberstyle=\tiny\color{gray},
  numbers=left,
  numbersep=8pt,
  stepnumber=1,
  showstringspaces=false,
  breaklines=true,
  frame=none,
  tabsize=4,
  captionpos=b,
  morekeywords={print, input, type, int, float, str, bool}
}

% ============================================================
%  EN-TÊTE ET PIED DE PAGE
% ============================================================
\pagestyle{fancy}
\fancyhf{}
\renewcommand{\headrulewidth}{0pt}
\fancyhead[L]{%
  \begin{tikzpicture}[remember picture, overlay]
    \fill[bleuPrimaire] (0,0) rectangle (\paperwidth, 1.1cm);
  \end{tikzpicture}%
  \color{white}\small\textbf{Cours Python — Mon Apprentissage}
}
\fancyhead[R]{\color{white}\small\textbf{Chapitre 1 : Introduction à Python}}
\fancyfoot[C]{%
  \begin{tikzpicture}[remember picture, overlay]
    \fill[bleuClair] (0,0) rectangle (\paperwidth, 0.7cm);
    \draw[bleuSecondaire, line width=1pt] (0, 0.7cm) -- (\paperwidth, 0.7cm);
  \end{tikzpicture}%
  \color{bleuPrimaire}\small\thepage
}

% ============================================================
%  STYLE DES TITRES
% ============================================================
\titleformat{\section}{%
  \color{bleuPrimaire}\Large\bfseries
}{}{0em}{}[\vspace{-0.3em}\textcolor{bleuSecondaire}{\rule{\linewidth}{1.5pt}}]

\titleformat{\subsection}{%
  \color{bleuSecondaire}\large\bfseries
}{}{0em}{}

\titlespacing{\section}{0pt}{1.5em}{0.8em}
\titlespacing{\subsection}{0pt}{1em}{0.5em}

% ============================================================
%  BOÎTES PERSONNALISÉES
% ============================================================
\newtcolorbox{boiteRetenir}{
  enhanced,
  breakable,
  colback=bleuClair,
  colframe=bleuSecondaire,
  arc=4pt,
  boxrule=1.5pt,
  left=10pt, right=10pt, top=8pt, bottom=8pt,
  title={\faLightbulb\quad À retenir},
  fonttitle=\bfseries\color{bleuPrimaire},
  attach boxed title to top left={yshift=-2mm, xshift=5mm},
  boxed title style={colback=bleuClair, colframe=bleuSecondaire}
}

\newtcolorbox{boiteNote}{
  enhanced,
  breakable,
  colback=orange!8,
  colframe=orangeNote,
  arc=4pt,
  boxrule=1.5pt,
  left=10pt, right=10pt, top=8pt, bottom=8pt,
  title={\faExclamationTriangle\quad Note importante},
  fonttitle=\bfseries\color{orangeNote},
  attach boxed title to top left={yshift=-2mm, xshift=5mm},
  boxed title style={colback=orange!8, colframe=orangeNote}
}

\newtcblisting{codeblock}[1]{
  enhanced,
  breakable,
  listing only,
  listing style=pythonstyle,
  colback=codebg,
  colframe=bleuSecondaire,
  arc=6pt,
  boxrule=1pt,
  title={\faCode\quad #1},
  fonttitle=\bfseries\color{white}\small,
  attach boxed title to top left={yshift=-2mm, xshift=5mm},
  boxed title style={colback=bleuSecondaire, colframe=bleuSecondaire},
  left=6pt, right=6pt, top=8pt, bottom=6pt
}

% ============================================================
%  PAGE DE TITRE
% ============================================================
\begin{document}

\begin{titlepage}
  \begin{tikzpicture}[remember picture, overlay]
    \fill[bleuPrimaire] (current page.north west) rectangle
      ([yshift=-10cm] current page.north east);
    \fill[bleuSecondaire] ([yshift=-10cm] current page.north west) rectangle
      ([yshift=-11cm] current page.north east);
    \fill[grisClair] ([yshift=-11cm] current page.north west) rectangle
      (current page.south east);
    \fill[bleuAccent, opacity=0.15]
      ([xshift=12cm, yshift=-5cm] current page.north west) circle (5cm);
    \fill[bleuAccent, opacity=0.08]
      ([xshift=15cm, yshift=-3cm] current page.north west) circle (3cm);
  \end{tikzpicture}

  \vspace*{2cm}

  \begin{center}
    {\fontsize{14}{16}\selectfont\color{bleuClair}\textbf{MON APPRENTISSAGE — SERIES PYTHON}}\\[0.5cm]
    {\fontsize{42}{48}\selectfont\color{white}\textbf{Python}}\\[0.2cm]
    {\fontsize{20}{24}\selectfont\color{bleuAccent}\textbf{Du zéro à l'expert}}\\[0.3cm]
    \textcolor{white!70}{\rule{8cm}{0.5pt}}\\[0.5cm]
    {\fontsize{16}{20}\selectfont\color{white}%
      \textbf{Chapitre 1 — Introduction à Python}}\\[0.2cm]
    {\fontsize{13}{16}\selectfont\color{bleuClair}%
      \textit{Le premier pas de mon apprentissage}}
  \end{center}

  \vspace{3.5cm}

  \begin{center}
    \begin{tikzpicture}
      \fill[bleuClair, rounded corners=8pt] (0,0) rectangle (12cm, 3.2cm);
      \draw[bleuSecondaire, line width=1.5pt, rounded corners=8pt]
        (0,0) rectangle (12cm, 3.2cm);
      \node[anchor=west] at (0.4, 2.5) {%
        \color{bleuPrimaire}\small\textbf{Auteur :}
      };
      \node[anchor=west] at (0.4, 1.9) {%
        \color{grisTexte}\large\textbf{KETOTSA AMÉVI CLAUDE}
      };
      \node[anchor=west] at (0.4, 1.2) {%
        \color{bleuPrimaire}\small\textbf{Date de publication :}
      };
      \node[anchor=west] at (0.4, 0.6) {%
        \color{grisTexte}\small\today
      };
    \end{tikzpicture}
  \end{center}

  \vfill
  \begin{center}
    \color{bleuSecondaire}\small\faLinkedin\quad
    \textit{Publié dans le cadre de mon parcours d'apprentissage Python}
  \end{center}
\end{titlepage}

% ============================================================
%  TABLE DES MATIÈRES
% ============================================================
\newpage
\tableofcontents
\newpage

% ============================================================
%  CHAPITRE 1
% ============================================================

\section*{\faPython\quad Chapitre 1 — Introduction à Python et à la Programmation}
\addcontentsline{toc}{section}{Chapitre 1 — Introduction à Python et à la Programmation}

\vspace{0.3cm}

\begin{boiteRetenir}
Ce chapitre marque le \textbf{premier pas} de mon parcours Python. Je résume ici les notions fondamentales que j'ai apprises : ce qu'est Python, comment l'installer, et comment écrire mon tout premier programme.
\end{boiteRetenir}

\vspace{0.5cm}

\subsection{Qu'est-ce que la programmation ?}

La programmation, c'est l'art de donner des instructions à un ordinateur pour qu'il exécute des tâches précises. Un programme, c'est une suite d'instructions écrites dans un \textbf{langage de programmation} que la machine peut comprendre et exécuter.

Imaginez une recette de cuisine : vous donnez des étapes claires, dans un ordre précis, et le résultat final apparaît. C'est exactement ce que fait un programmeur — sauf qu'au lieu de faire un gâteau, on automatise des tâches, on traite des données, ou on construit des applications.

\subsection{Pourquoi Python ?}

Python est aujourd'hui l'un des langages de programmation les plus populaires au monde, et pour de très bonnes raisons :

\begin{itemize}[leftmargin=1.5em, itemsep=4pt]
  \item[\textcolor{bleuSecondaire}{\faCheckCircle}]
    \textbf{Lisibilité} — La syntaxe de Python ressemble à de l'anglais naturel.
  \item[\textcolor{bleuSecondaire}{\faCheckCircle}]
    \textbf{Polyvalence} — Web, data science, IA, automatisation, jeux\ldots
  \item[\textcolor{bleuSecondaire}{\faCheckCircle}]
    \textbf{Communauté} — Des millions de développeurs, des milliers de bibliothèques.
  \item[\textcolor{bleuSecondaire}{\faCheckCircle}]
    \textbf{Gratuit et open source} — Accessible à tous, partout dans le monde.
\end{itemize}

\subsection{Histoire de Python}

Python a été créé par \textbf{Guido van Rossum}, un développeur néerlandais, et publié pour la première fois en \textbf{1991}. Le nom vient non pas du serpent, mais de la troupe comique britannique \textit{Monty Python's Flying Circus}, dont Guido était fan.

Aujourd'hui, Python est géré par la \textbf{Python Software Foundation} et en est à sa version \textbf{3.x}, qui est la version moderne et active.

\begin{boiteNote}
Toujours utiliser \textbf{Python 3} (jamais Python 2, qui n'est plus maintenu depuis 2020).
\end{boiteNote}

\subsection{Installation de Python}

\subsubsection*{Étape 1 — Télécharger Python}
Rendez-vous sur \href{https://www.python.org}{\textcolor{bleuSecondaire}{\textbf{python.org}}} et téléchargez la dernière version stable.

\subsubsection*{Étape 2 — Choisir un éditeur de code}

\begin{center}
\begin{tikzpicture}
  \foreach \nom/\desc/\pos in {
    {VS Code}/{Léger, gratuit, très populaire}/0,
    {PyCharm}/{Puissant, dédié Python}/4.5,
    {Jupyter}/{Idéal pour la data science}/9
  }{
    \fill[bleuClair, rounded corners=5pt] (\pos, 0) rectangle (\pos+4, 2);
    \draw[bleuSecondaire, line width=1pt, rounded corners=5pt]
      (\pos, 0) rectangle (\pos+4, 2);
    \node[font=\bfseries\small, text=bleuPrimaire] at (\pos+2, 1.4) {\nom};
    \node[font=\tiny, text=grisTexte, align=center] at (\pos+2, 0.7) {\desc};
  }
\end{tikzpicture}
\end{center}

\vspace{0.3cm}

\subsubsection*{Étape 3 — Vérifier l'installation}

\begin{codeblock}{Terminal — Vérification}
python --version
# Resultat attendu : Python 3.x.x
\end{codeblock}

\subsection{Mon premier programme Python}

La tradition en programmation veut que le tout premier programme affiche le message \textbf{"Hello, World!"}. En Python, c'est d'une simplicité remarquable :

\begin{codeblock}{hello.py — Mon tout premier programme}
print("Hello, World!")
\end{codeblock}

\begin{center}
\begin{tikzpicture}
  \fill[vertCode!10, rounded corners=6pt] (0,0) rectangle (10, 1.2);
  \draw[vertCode, line width=1pt, rounded corners=6pt] (0,0) rectangle (10, 1.2);
  \node[font=\ttfamily\small, text=vertCode] at (5, 0.6)
    {\faTerminal\quad Résultat : Hello, World!};
\end{tikzpicture}
\end{center}

\vspace{0.3cm}

La fonction \texttt{print()} est l'une des fonctions les plus utilisées en Python. Elle permet d'\textbf{afficher du texte ou des données} dans le terminal.

\subsection{La philosophie Python — Le Zen de Python}

Python a une philosophie officielle, accessible en tapant \texttt{import this} dans Python :

\begin{codeblock}{import this — La philosophie Python}
import this
# Affiche le "Zen of Python" de Tim Peters
\end{codeblock}

Quelques principes clés à retenir :

\begin{itemize}[leftmargin=1.5em, itemsep=4pt]
  \item[\textcolor{bleuAccent}{\faQuoteLeft}]
    \textit{"Beautiful is better than ugly."} — Le code doit être beau et lisible.
  \item[\textcolor{bleuAccent}{\faQuoteLeft}]
    \textit{"Simple is better than complex."} — Toujours privilégier la simplicité.
  \item[\textcolor{bleuAccent}{\faQuoteLeft}]
    \textit{"Readability counts."} — La lisibilité du code est une priorité.
\end{itemize}

\subsection{Les conventions PEP 8}

\textbf{PEP 8} est le guide de style officiel de Python. Il définit comment écrire du code propre et lisible :

\begin{itemize}[leftmargin=1.5em, itemsep=4pt]
  \item[\textcolor{bleuSecondaire}{\faArrowRight}]
    Utiliser \textbf{4 espaces} pour l'indentation (jamais des tabulations).
  \item[\textcolor{bleuSecondaire}{\faArrowRight}]
    Limiter les lignes à \textbf{79 caractères} maximum.
  \item[\textcolor{bleuSecondaire}{\faArrowRight}]
    Nommer les variables en \textbf{minuscules\_avec\_underscore}.
  \item[\textcolor{bleuSecondaire}{\faArrowRight}]
    Ajouter des \textbf{commentaires} pour expliquer le code.
\end{itemize}

\begin{codeblock}{Exemple de bon style Python (PEP 8)}
# Ceci est un commentaire — il explique le code
mon_prenom = "Alice"       # Variable bien nommée
age_utilisateur = 25       # Minuscules + underscore

print("Bonjour", mon_prenom)
print("Vous avez", age_utilisateur, "ans")
\end{codeblock}

\subsection{Résumé du Chapitre 1}

\begin{tcolorbox}[
  enhanced,
  breakable,
  colback=bleuPrimaire,
  colframe=bleuPrimaire,
  arc=8pt,
  left=15pt, right=15pt, top=10pt, bottom=10pt
]
  \color{white}
  \textbf{\large\faClipboardList\quad Ce que j'ai appris dans ce chapitre :}

  \vspace{0.5em}
  \begin{itemize}[leftmargin=1.5em, itemsep=6pt]
    \item[\textcolor{bleuAccent}{\faCheckCircle}]
      La programmation, c'est donner des instructions précises à un ordinateur.
    \item[\textcolor{bleuAccent}{\faCheckCircle}]
      Python est un langage lisible, polyvalent, gratuit et très populaire.
    \item[\textcolor{bleuAccent}{\faCheckCircle}]
      Python a été créé par Guido van Rossum en 1991 — toujours utiliser Python 3.
    \item[\textcolor{bleuAccent}{\faCheckCircle}]
      \texttt{print()} est la première fonction à connaître pour afficher du texte.
    \item[\textcolor{bleuAccent}{\faCheckCircle}]
      La PEP 8 guide l'écriture d'un code propre et professionnel.
  \end{itemize}

  \vspace{0.5em}
  \textit{\textcolor{bleuClair}{"Le plus long voyage commence par un premier pas." — Lao Tseu}}
\end{tcolorbox}

\vspace{1cm}
\begin{center}
  \begin{tikzpicture}
    \fill[bleuClair, rounded corners=6pt] (0,0) rectangle (16, 1.5);
    \draw[bleuSecondaire, line width=1pt, rounded corners=6pt]
      (0,0) rectangle (16, 1.5);
    \node[font=\small, text=bleuPrimaire] at (8, 0.75) {%
      \faLinkedin\quad \textbf{KETOTSA AMÉVI CLAUDE} — Mon apprentissage Python   
      \quad|\quad Chapitre 1 sur 34
    };
    Chapitre 1 sur 34
    \today
  \end{tikzpicture}
\end{center}

\end{document}

